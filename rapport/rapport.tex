\documentclass[a4paper, 10pt, french]{article}
% Préambule; packages qui peuvent être utiles
   \RequirePackage[T1]{fontenc}        % Ce package pourrit les pdf...
   \RequirePackage{babel,indentfirst}  % Pour les césures correctes,
                                       % et pour indenter au début de chaque paragraphe
   \RequirePackage[utf8]{inputenc}   % Pour pouvoir utiliser directement les accents
                                     % et autres caractères français
   \RequirePackage{lmodern,tgpagella} % Police de caractères
   \textwidth 17cm \textheight 25cm \oddsidemargin -0.24cm % Définition taille de la page
   \evensidemargin -1.24cm \topskip 0cm \headheight -1.5cm % Définition des marges
   \RequirePackage{latexsym}                  % Symboles
   \RequirePackage{amsmath}                   % Symboles mathématiques
   \RequirePackage{tikz}   % Pour faire des schémas
   \RequirePackage{graphicx} % Pour inclure des images
   \RequirePackage{listings} % pour mettre des listings
% Fin Préambule; package qui peuvent être utiles

\title{Rapport de TP 4MMAOD : Génération de patch optimal}
\author{
Kamowski Nicolas  (groupe IF$_1$) 
Maubert Jonas  (groupe IF$_1$) 
}

\begin{document}

\maketitle


%%%%%%%%%%%%%%%%%%%%%%%%%%%%%%%%%%%%%%%%%%%%%%
\section{Principe de notre  programme (1 point)}
{\em Nous avons implémenté une méthode itérative de Bellman.
Nous avons choisis comme ordre de mettre en premier les destructions simples, puis les destructions multiples, en 3ème les ajouts et en derniers les substitutions. 
 Mettre ici une explication brève du principe de votre programme en  précisant la méthode implantée (récursive, itérative) et les
choix effectués (notamment pour l'ordonnancement des instructions).
} 

%%%%%%%%%%%%%%%%%%%%%%%%%%%%%%%%%%%%%%%%%%%%%%
\section{Analyse du coût théorique (3 points)}
{\em Donner ici l'analyse du coût théorique de votre programme en fonction des nombres $n_1$ et $n_2$ de lignes 
et $c_1$ et $c_2$ de caractères des deux fichiers en entrée.
 Pour chaque coût, donner la formule qui le caractérise (en justifiant brièvement pourquoi cette formule correspond à votre programme), 
 puis l'ordre du coût en fonction de $n_1, n_2, c_1, c_2$ en notation $\Theta$ de préférence, sinon $O$.}

  \subsection{Nombre  d'opérations en pire cas\,: }
    \paragraph{Justification\,: }
    {\em La justification peut être par exemple de la forme: \\ 
       "Le programme itératif contient les boucles $k_1=...$, $k_2= ...$ etc correspondant à la somme 
      $$T(n_1, n_2, c_1, c_2) = \sum_{k_1=...}^{...} ... \sum ... + \sum_{i=...}^{...} ...$$ 
      somme que nous avons calculée (ou majorée) par la technique de  ... " \\
      ou  encore\,:  \\
      "les appels récursifs du programme permettent de modéliser son coût par le système d'équations aux récurrences 
      $$T(k_1, k_2) = ...  \mbox{~avec~les~conditions~initiales~....~} $$
      Le coût indiqué est obtenu en résolvant ce système par la méthode de  .... "
    } 
  \subsection{Place mémoire requise\,: }
    \paragraph{Justification\,: }

  \subsection{Nombre de défauts de cache sur le modèle CO\,: }
    \paragraph{Justification\,: }


%%%%%%%%%%%%%%%%%%%%%%%%%%%%%%%%%%%%%%%%%%%%%%
\section{Compte rendu d'expérimentation (2 points)}
  \subsection{Conditions expérimentaless}
     {\em Décrire les conditions permettant la reproductibilité des mesures: on demande la description
      de la machine et la méthode utilisée pour mesurer le temps.
      Nous avons effectué les mesures sur 
      Pour calculer le temps nous avons utilisé la commande time, et pris le team réel.
     }

    \subsubsection{Description synthétique de la machine\,:} 
      {\em indiquer ici le  processeur et sa fréquence, la mémoire, le système d'exploitation. 
       Préciser aussi si la machine était monopolisée pour un test, ou notamment si 
       d'autres processus ou utilisateurs étaient en cours d'exécution. 
       Situation : ordinateur de la salle E103.
       Tests effectués avec firefox en cours d'éxecution pour noter les résultats.
      } 

    \subsubsection{Méthode utilisée pour les mesures de temps\,: } 
      {\em 
        Nous avons utilisé la commande ``time'' pour mesurer le temps de notre fonction. Les temps donnés sont en secondes.
Pour la moyenne, le même test est fait 5 fois de suite.
      }

  \subsection{Mesures expérimentales}
    {\em Le tableau suivant contient les temps d'exécution mesurés pour chacun des 6 benchmarks imposés
              (temps minimum, maximum et moyen sur 5 exécutions)
    }

    \begin{figure}[h]
      \begin{center}
        \begin{tabular}{|l||r||r|r|r||}
          \hline
          \hline
            & coût         & temps     & temps   & temps \\
            & du patch     & min       & max     & moyen \\
          \hline
          \hline
            benchmark1 &2540&   0,122   &0,133     &0,125     \\
          \hline
            benchmark2 &3120&0,429      &0,436     &0,431     \\
          \hline
            benchmark3 &809&0,871     &0,913     &0,884     \\
          \hline
            benchmark4 &1708&1,247     &1,275     &1,265     \\
          \hline
            benchmark5 &7553&2,337     &2,465     &2,387     \\
          \hline
            benchmark6 &37027&9,334     &9,440     &9,373     \\
          \hline
          \hline
        \end{tabular}
        \caption{Mesures des temps minimum, maximum et moyen de 5 exécutions pour les 6 benchmarks.}
        \label{table-temps}
      \end{center}
    \end{figure}

\subsection{Analyse des résultats expérimentaux}
{\em Donner  une réponse justifiée  à la question\,: 
              les  temps mesurés correspondent ils  à votre analyse théorique (nombre d’opérations et défauts de cache) ?
}

%%%%%%%%%%%%%%%%%%%%%%%%%%%%%%%%%%%%%%%%%%%%%%
\section{Question\,: et  si le coût d'un patch était sa taille en octets ? (1 point)}
{\em Préciser le principe de la résolution choisie (parmi celles vues en cours); donner  les modifications à apporter (soit à vos  équations, soit à votre programme, au choix) 
pour s'adapter à cette nouvelle fonction de coût. 
}

\end{document}
%% Fin mise au format

